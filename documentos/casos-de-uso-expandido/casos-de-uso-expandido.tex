\documentclass[12pt,a4paper,onecolumn,titlepage]{article}

\usepackage[brazil]{babel}
%\usepackage[latin1]{inputenc}
\usepackage[utf8]{inputenc}
\usepackage{graphicx}

\begin{document} %Inicio do documento

\begin{titlepage} %Capa
	
	\vfill
	\begin{center}
	
		{\large \textbf{Faculdade de Ciências e Tecnologia\\Universidade Estadual Paulista\\``Júlio de Mesquita Filho''}} \\[3cm]
		{\large \textbf{Bruno Santos de Lima}}\\
		{\large \textbf{Leandro Ungari Cayres}}\\[4cm]
		{\Large Casos de Uso Expandido}\\
		{\Large Sistema LearnStation}\\[4cm]

	\hspace{.45\textwidth} %posiciona a minipage
	\begin{minipage}{.5\textwidth}
		\large Disciplina de Engenharia de Software I. Professor Dr. Rogério Eduardo Garcia.\\[0.5cm]
	\end{minipage}

	\vfill
	\vspace{1.5cm}
	
	\large \textbf{Presidente Prudente\\}
	\large \textbf{Maio - 2016}
	
	\end{center}
	
\end{titlepage}
%Fim da capa

%Conteúdo do documento
\renewcommand{\baselinestretch}{1.1}
%%%%%%%%%%%%%%%1


\begin{table}[h!]
\begin{center}
\begin{tabular}{p{2.5cm} p{9.5cm}}
Caso de uso: & \textbf{Criar Conta de Usuário} \\ \hline
Atores: & Usuário \\ \hline
Finalidade: & Capturar a inscrição de um novo usuário. \\ \hline
Tipo: & Primário e Essencial\\ \hline
Visão geral: & O Usuário acessa a seção de cadastro fornecendo informações pessoais. Em seguida, os dados são validados e o cadastro concluído. \\ \hline
Referências Cruzadas: & Requisito: RF 1.1

%Requisito 1.1	
\end{tabular}
\end{center}
\end{table}


\begin{center}
\def\arraystretch{1.1}
\begin{tabular}{|p{6cm}|p{6cm}|}

\hline
\textbf{Ação do Ator} & \textbf{Resposta do Sistema} \\ \hline
\textbf{1.} O Usuário acessa a seção de cadastro e fornece as suas informações pessoais. & \textbf{2.} Cada informação fornecida é validada. \\ \hline
\textbf{3.} Por fim, o usuário confirma o seu cadastro. & \textbf{4.} A criação da conta de usuário é efetivada. \\ \hline
\end{tabular}
\end{center}

\textbf{Sequências Alternativas:} N/A

\newpage

\begin{table}[h!]
\begin{center}
\begin{tabular}{p{2.5cm} p{9.5cm}}
Caso de uso: & \textbf{Iniciar sessão de usuário} \\ \hline
Atores: & Usuário \\ \hline
Finalidade: & Permitir o acesso à conta do respectivo usuário. \\ \hline
Tipo: & Primário e Essencial \\ \hline
Visão geral: & O Usuário acessa a seção de entrada do sistema, fornece os dados de entrada que são validados pelo sistema e em seguida tem acesso ao seu perfil. \\ \hline
Referências Cruzadas: & Requisito: RF. 1.17

%Requisito 1.17	
\end{tabular}
\end{center}
\end{table}

\begin{center}
\def\arraystretch{1.1}
\begin{tabular}{|p{6cm}|p{6cm}|}

\hline
\textbf{Ação do Ator} & \textbf{Resposta do Sistema} \\ \hline
\textbf{1.} O Usuário acessa a seção de entrada e fornece as suas informações pessoais. & \textbf{2.} Cada informação fornecida é validada. \\ \hline
 & \textbf{3.} O Uusuário é direcionado ao seu perfil. \\ \hline
\end{tabular}
\end{center}

\textbf{Sequências Alternativas:} N/A

\newpage

\begin{table}[h!]
\begin{center}
\begin{tabular}{p{2.5cm} p{9.5cm}}
Caso de uso: & \textbf{Finalizar sessão de usuário} \\ \hline
Atores: & Usuário \\ \hline
Finalidade: & Permitir finalizar a sessão da conta do respectivo usuário. \\ \hline
Tipo: & Primário e Essencial \\ \hline
Visão geral: & O Usuário dentro do sistema, pode aciona a finalização de sua sessão através de qualquer página. \\ \hline
Referências Cruzadas: & Requisito: RF. 1.17

%Requisito 1.17	
\end{tabular}
\end{center}
\end{table}

\begin{center}
\def\arraystretch{1.1}
\begin{tabular}{|p{6cm}|p{6cm}|}

\hline
\textbf{Ação do Ator} & \textbf{Resposta do Sistema} \\ \hline
\textbf{1.} O Usuário requisita a finalização de sua sessão. & \textbf{2.} A conta é finalizada e redireciona-se para outra página de acesso. \\ \hline
\end{tabular}
\end{center}

\textbf{Sequências Alternativas:} N/A

\newpage
%%%%%%%%%%%%%%%%%%2


\begin{table}[h!]
\begin{center}
\begin{tabular}{p{2.5cm} p{9.5cm}}
Caso de uso: & \textbf{Listar Cursos e Usuários} \\ \hline
Finalidade: & Exibir uma lista contendo todos os usuários e/ou cursos correspondentes a uma dada entrada.\\ \hline
Atores: & Usuário \\ \hline
Tipo: & Primário e Essencial\\ \hline
Visão geral: & O Usuário, por meio do mecanismo de pesquisa, de forma textual busca cursos ou usuários através de seus respectivos nome de usuário ou curso, e área de conhecimento para cursos. Desta forma, tendo acesso ao conteúdo do perfil do usuário ou curso buscado.\\ \hline
Referências Cruzadas: & Requisitos: RF 1.2, RF 1.3, RF 1.6 e RF 2.2

%Requisito 1.2, 1.3, 1.6, 2.2
\end{tabular}
\end{center}
\end{table}


\begin{center}
\def\arraystretch{1.1}
\begin{tabular}{|p{6cm}|p{6cm}|}

\hline
\textbf{Ação do Ator} & \textbf{Resposta do Sistema} \\ \hline
\textbf{1.} O Usuário busca o curso ou outro usuário desejado pelo respectivo nome. & \textbf{2.} Uma lista é exibida, separado por categorias, dos resultados compatíveis com a entrada fornecida. \\ \hline
\end{tabular}
\end{center}

\textbf{Sequências Alternativas:}\\
Linha 2: Caso não sejam encontrados resultados compatíveis, um indicativo de nenhum resultado é mostrado.
\newpage

%%%%%%%%%%%%%%3

\begin{table}[h!]
\begin{center}
\begin{tabular}{p{2.5cm} p{9.5cm}}
Caso de uso: & \textbf{Enviar Solicitação de Amizade} \\ \hline
Atores: & Usuário \\ \hline
Finalidade: & Permitir a requsição de amizade entre usuários. \\ \hline
Tipo: & Primário e Essencial \\ \hline
Visão geral: & O Usuário busca pelo referido usuário, em seguida acessa o perfil deste e envia uma solicitação de amizade. \\ \hline
Referências Cruzadas: & Requisito: RF 1.4 \\ & Caso de uso: Listar Cursos e Usuários.
%Requisito 1.4
\end{tabular}
\end{center}
\end{table}


\begin{center}
\def\arraystretch{1.1}
\begin{tabular}{|p{6cm}|p{6cm}|}

\hline
\textbf{Ação do Ator} & \textbf{Resposta do Sistema} \\ \hline
\textbf{1.} O Usuário seleciona o usuário específico. & \textbf{2.} O perfil do usuário é exibido. \\ \hline
\textbf{3.} O Usuário realiza uma pedido de amizade. & \textbf{4.} Uma solicitação é enviada ao usuário correspondente. \\ \hline
\end{tabular}
\end{center}


\textbf{Sequências Alternativas:} N/A
\newpage

%%%%%%%%%%%%%%%4

\begin{table}[h!]
\begin{center}
\begin{tabular}{p{2.5cm} p{9.5cm}}
Caso de uso: & \textbf{Responder Solicitação} \\ \hline
Atores: & Usuário \\ \hline
Finalidade: & Possibilita a confirmação ou não do pedido de amizade. \\ \hline
Tipo: & Primário e Essencial\\ \hline
Visão geral: & O Usuário após acessar sua conta, verifica a existência de uma notificação, a qual é uma solicitação de amizade, em seguida, se desejado o Usuário poderá confirmar, recusar ou ignorar.\\ \hline
Referências Cruzadas: & Requisitos: RF 1.4 e RF 1.13 
%Requisito 1.4, 1.13
\end{tabular}
\end{center}
\end{table}


\begin{center}
\def\arraystretch{1.1}
\begin{tabular}{|p{6cm}|p{6cm}|}

\hline
\textbf{Ação do Ator} & \textbf{Resposta do Sistema} \\ \hline
\textbf{1.} O Usuário acessa o seu perfil. & \textbf{2.} As solicitações de amizade referentes a este usuário são enviadas para o perfil. \\ \hline
\textbf{3. } Para cada, o usuário pode aceitar, ignorar ou cancelar a solicitação. & \textbf{4.} Para cada usuário solicitante é enviada a resposta da solicitação conforme a escolha o usuário requisito. \\ \hline
\end{tabular}
\end{center}

\textbf{Sequências Alternativas:}\\
Linha 4: Caso a solicitação de amizade seja ignorada, o usuário requisitante não receberá resposta.

\newpage

%%%%%%%%%5

\begin{table}[h!]
\begin{center}
\begin{tabular}{p{2.5cm} p{9.5cm}}
Caso de uso: & \textbf{Cancelar amizade} \\ \hline
Atores: & Usuário \\ \hline
Finalidade: & Permite desfazer a relação de amizade entre dois usuários. \\ \hline
Tipo: & Primário e Essencial\\ \hline
Visão geral: & O Usuário realiza uma busca pelo referido amigo, ou acessa a lista de amizades, em seguida, visita o perfil do determinado amigo, efetua o cancelamento desta amizade. \\ \hline
Referências Cruzadas: & Requisito: RF 1.5 \\ & Caso de uso: Listar Cursos e Usuários.

%Requisito 1.5
\end{tabular}
\end{center}
\end{table}


\begin{center}
\def\arraystretch{1.1}
\begin{tabular}{|p{6cm}|p{6cm}|}

\hline
\textbf{Ação do Ator} & \textbf{Resposta do Sistema} \\ \hline
\textbf{1.} O Usuário seleciona o usuário específico. & \textbf{2.} O perfil do usuário é exibido. \\ \hline
\textbf{3.} O Usuário realiza o cancelamento da amizade. & \textbf{4.} A relação de amizade é removida do registro.\\ \hline
\end{tabular}
\end{center}
\textbf{Sequências Alternativas:} N/A
\newpage



%%%%%%%%%%6

\begin{table}[h!]
\begin{center}
\begin{tabular}{p{2.5cm} p{9.5cm}}
Caso de uso: & \textbf{Efetuar Inscrição em Curso} \\ \hline
Atores: & Aluno \\ \hline
Finalidade: & Possibilita que o usuário se inscreva no curso de sua escolha. \\ \hline
Tipo: & Primário e Essencial\\ \hline
Visão geral: & Após a realização da busca pelo referido curso e visualizar da descrição do curso, o Aluno realiza a inscrição neste curso. \\ \hline
Referências Cruzadas: & Requisito: RF 1.7 \\ & Caso de uso: Listar Cursos e Usuários.

%Requisito 1.7
\end{tabular}
\end{center}
\end{table}


\begin{center}
\def\arraystretch{1.1}
\begin{tabular}{|p{6cm}|p{6cm}|}

\hline
\textbf{Ação do Ator} & \textbf{Resposta do Sistema} \\ \hline
\textbf{1.} O Usuário seleciona o curso específico. & \textbf{2.} A descrição do curso é exibida. \\ \hline
\textbf{3.} O Usuário realiza a inscrição no curso. & \textbf{4.} Registra o curso na lista de cursos do usuário. \\ \hline
\end{tabular}
\end{center}
\textbf{Sequências Alternativas:} N/A
\newpage


%%%%%%%%%% 7

\begin{table}[h!]
\begin{center}
\begin{tabular}{p{2.5cm} p{9.5cm}}
Caso de uso: & \textbf{Cancelar Inscrição em Curso} \\ \hline
Atores: & Aluno \\ \hline
Finalidade: & Permite o encerramento da inscrição do curso, bloqueando o acesso ao conteúdo deste. \\ \hline
Tipo: & Primário e Essencial\\ \hline
Visão geral: & O Aluno realiza uma busca pelo curso, ou acessa a lista de cursos em que está inscrito, em seguida, acessa a descrição deste e realiza o cancelamento da sua inscrição. \\ \hline
Referências Cruzadas: & Requisito: RF 1.8 \\ & Caso de uso: Listar Cursos e Usuários.
%Requisito 1.8
\end{tabular}
\end{center}
\end{table}


\begin{center}
\def\arraystretch{1.1}
\begin{tabular}{|p{6cm}|p{6cm}|}

\hline
\textbf{Ação do Ator} & \textbf{Resposta do Sistema} \\ \hline
\textbf{1.} O Usuário seleciona o curso específico. & \textbf{2.} A descrição do curso é exibida. \\ \hline
\textbf{3.} O Usuário realiza o cancelamento da inscrição no curso. & \textbf{4.} O curso é removido da lista do usuário. \\ \hline
\end{tabular}
\end{center}
\textbf{Sequências Alternativas:} N/A

\newpage


%%%%%%%%%%%8

\begin{table}[h!]
\begin{center}
\begin{tabular}{p{2.5cm} p{9.5cm}}
Caso de uso: & \textbf{Criar Novo Curso} \\ \hline
Atores: & Professor \\ \hline
Finalidade: & Permite que um professor disponibilize aulas sobre uma determinada temática. \\ \hline
Tipo: & Primário e Essencial\\ \hline
Visão geral: & O Professor acessa a seção de cadastro de curso, fornecendo as informações sobre o curso que será criado e finaliza a operação. \\ \hline
Referências Cruzadas: & Requisito: RF 1.9

%Requisito 1.9
\end{tabular}
\end{center}
\end{table}



\begin{center}
\def\arraystretch{1.1}
\begin{tabular}{|p{6cm}|p{6cm}|}

\hline
\textbf{Ação do Ator} & \textbf{Resposta do Sistema} \\ \hline
\textbf{1.} O Professor acessa a seção de administração de curso. &  \\ \hline
\textbf{2.} O Professor adiciona um novo curso com as informações relativas. & \textbf{3.} Essas informações são armazenadas e é requerida a confirmação. \\ \hline
\textbf{4.} O Professor realiza a confirmação de novo curso. & \textbf{5.} O novo curso é registrado. \\ \hline
\end{tabular}
\end{center}
\textbf{Sequências Alternativas:} N/A
\newpage

%-------------------------------------------------------------------------------------------------------------
%Avaliação de curso
\begin{table}[h!]
\begin{center}
\begin{tabular}{p{2.5cm} p{9.5cm}}
Caso de uso: & \textbf{Avaliação de curso} \\ \hline
Atores: & Aluno \\ \hline
Finalidade: & Capturar a avaliação de um curso\\ \hline
Tipo: & Primário e Essencial \\ \hline
Visão geral: & O Aluno inscrito no curso em questão acessa este por meio da sua lista de cursos inscritos, ou por meio da busca de cursos, em seguida, avalia este atribuindo uma nota. \\ \hline
Referências Cruzadas: & Requisitos: RF 1.6 e RF 1.12\\

%Requisito 1.12
\end{tabular}
\end{center}
\end{table} 

\begin{center}
\def\arraystretch{1.1}
\begin{tabular}{|p{6cm}|p{6cm}|}

\hline
\textbf{Ação do Ator} & \textbf{Resposta do Sistema} \\ \hline
\textbf{1.} Este caso de uso tem inicio quando o Aluno acessa a descrição de um determinado curso  & \textbf{2.} Exibe a descrição do curso para o Aluno  \\ \hline
\textbf{3.} O Aluno acessa a seção de avaliação do curso em questão.  & \textbf{4.} Exibe a seção de avaliação do curso, descrevendo a nota atual deste curso e permitindo ao Aluno atibuir uma nota ao curso em questão  \\ \hline
\textbf{5.} O Aluno atribui uma nota para o curso em questão  &   \\ \hline
\textbf{6.} O Aluno finaliza a avaliação do curso  & \textbf{7.} A nota do Aluno é resgistada e é calculada um nova nota geral do curso avalidado \\ \hline
\end{tabular}
\end{center}

\textbf{Sequências Alternativas:} N/A


\newpage

%-------------------------------------------------------------------------------------------------------------
%Enviar mensagem para um amigo
\begin{table}[h!]
\begin{center}
\begin{tabular}{p{2.5cm} p{9.5cm}}
Caso de uso: & \textbf{Enviar mensagem para um amigo} \\ \hline
Atores: & Remetente, Destinatário \\ \hline
Finalidade: & Registrar e enviar mensagem de um remetente para um destinatário\\ \hline
Tipo: & Primário e Essencial \\ \hline
Visão geral: & O Usuário por meio de uma lista de amigos, selecione um referido amigo e envia mensagens textuais através de uma seção de conversa. \\ \hline
Referências Cruzadas: & Requisitos: RF 1.2 e RF 1.12\\

%Requisito 1.12
\end{tabular}
\end{center}
\end{table} 

\begin{center}
\def\arraystretch{1.1}
\begin{tabular}{|p{6cm}|p{6cm}|}

\hline
\textbf{Ação do Ator} & \textbf{Resposta do Sistema} \\ \hline
\textbf{1.} Este caso de uso tem inicio quando o Remetente seleciona um Destinatário em sua lista de amigos para iniciar uma conversa  &  \\ \hline
\textbf{2.} O Remetente escreve a mensagem  &   \\ \hline
\textbf{3.} O Remetente envia a mensagem para o Destinatário  &
\textbf{4.} Envia a mensagem do Remetente e também envia uma notificação de uma nova mensagem recebida para o Destinatário \\ \hline
\end{tabular}
\end{center}

\textbf{Sequências Alternativas:} N/A

\newpage

%-------------------------------------------------------------------------------------------------------------
%Receber mensagem de um amigo
\begin{table}[h!]
\begin{center}
\begin{tabular}{p{2.5cm} p{9.5cm}}
Caso de uso: & \textbf{Receber mensagem de um amigo} \\ \hline
Atores: & Destinatário, Remetente \\ \hline
Finalidade: & Notificar e encaminhar mensagem para o destinatário\\ \hline
Tipo: & Primário e Essencial \\ \hline
Visão geral: & O Usuário acessa sua conta, verifica a existência de notificações, no caso, de recebimento de mensagens, em seguida, seleciona uma determinada notificação tendo acesso ao conteúdo da mensagem. \\ \hline
Referências Cruzadas: & Requisitos: RF 1.12 e RF 1.13\\

%Requisito 1.12, 1.13
\end{tabular}
\end{center}
\end{table} 

\begin{center}
\def\arraystretch{1.1}
\begin{tabular}{|p{6cm}|p{6cm}|}

\hline
\textbf{Ação do Ator} & \textbf{Resposta do Sistema} \\ \hline
\textbf{1.} Este caso de uso tem inicio quando o Destinatário verifica suas notificações de mensagens enviadas  &  \\ \hline
\textbf{2.} O destinatário acessa o conteúdo desta mensagem  & 
\textbf{3.} Informa ao destinatário o conteúdo da mensagem no qual ele recebeu \\ \hline
\textbf{4.} Destinatário lê o conteúdo da mensagem recebida. & \\ \hline
\end{tabular}
\end{center}

\textbf{Sequências Alternativas:} N/A

\newpage

%-------------------------------------------------------------------------------------------------------------
%Encerrar conta de usuário
\begin{table}[h!]
\begin{center}
\begin{tabular}{p{2.5cm} p{9.5cm}}
Caso de uso: & \textbf{Encerrar conta de usuário} \\ \hline
Atores: & Usuário \\ \hline
Finalidade: & Finalizar e encerrar um conta de usuário\\ \hline
Tipo: & Primário e Essencial \\ \hline
Visão geral: & O Usuário acessa sua conta, em seguida, solicita o cancelamento da sua conta fornecendo informação de segurança. \\ \hline
Referências Cruzadas: & Requisitos: RF 1.14 e RF 1.15\\

%Requisito 1.14, 1.15
\end{tabular}
\end{center}
\end{table} 

\begin{center}
\def\arraystretch{1.1}
\begin{tabular}{|p{6cm}|p{6cm}|}

\hline
\textbf{Ação do Ator} & \textbf{Resposta do Sistema} \\ \hline
\textbf{1.} Este caso de uso tem inicio quando o Usuário acessa seu perfil e solicita a exclusão de sua conta  & \\ \hline
\textbf{2.} O Usuário envia a senha de sua conta  & \textbf{3.} Recebe a senha e verifica se a mesma está correta  \\ \hline
\textbf{4.} O Usuário confirma o encerramento de sua conta  & \textbf{5.} Encerra a conta do Usuário de forma definitiva. \\ \hline
\end{tabular}
\end{center}

\textbf{Sequências Alternativas:} \\
\textbf{Linha 3:} Ao verificar que a senha fornecida pelo Usuário está incorreta, informa o mesmo sobre a senha incorreta e não exclui a conta do Usuário.

\newpage

%-------------------------------------------------------------------------------------------------------------
%Denunciar Conteúdo Inadequado
\begin{table}[h!]
\begin{center}
\begin{tabular}{p{2.5cm} p{9.5cm}}
Caso de uso: & \textbf{Denunciar Conteúdo Inadequado} \\ \hline
Atores: & Usuário \\ \hline
Finalidade: & Permitir e registrar denuncia de conteúdo inadequado\\ \hline
Tipo: & Primário e Essencial \\ \hline
Visão geral: & O Usuário ao acessar o perfil de um usuário ou a descrição de um curso, ao perceber a existência de conteúdo inadequado realizará, o usuário poderá denunciar ao administrador do sistema. \\ \hline
Referências Cruzadas: & Requisitos: RF 1.3, RF 1.16 e RF 2.2\\

%Requisito 1.16
\end{tabular}
\end{center}
\end{table} 


\begin{center}
\def\arraystretch{1.1}
\begin{tabular}{|p{6cm}|p{6cm}|}

\hline
\textbf{Ação do Ator} & \textbf{Resposta do Sistema} \\ \hline
\textbf{1.} Este caso de uso tem inicio quando o Usuário identifica um conteúdo inadequado & \\ \hline
\textbf{2.} O Usuário faz a denuncia do conteúdo  & \\ \hline
\textbf{3.} O Usuário informa a categoria e a descrição do conteúdo inadequado & \\ \hline
\textbf{4.} O Usuário envia a denuncia do conteúdo Inadequado & \textbf{5} Regrista a denuncia \\ \hline
\end{tabular}
\end{center}

\textbf{Sequências Alternativas:} N/A

\newpage

%-------------------------------------------------------------------------------------------------------------
%Adicionar Aula
\begin{table}[h!]
\begin{center}
\begin{tabular}{p{2.5cm} p{9.5cm}}
Caso de uso: & \textbf{Adicionar Aula} \\ \hline
Atores: & Professor \\ \hline
Finalidade: & Registrar nova aula inserida por um professor em um determinado curso\\ \hline
Tipo: & Primário e Essencial \\ \hline
Visão geral: & O Professor acessa a seção de administração de um curso, adiciona uma nova aula fornecendo seu respectivo conteúdo. \\ \hline
Referências Cruzadas: & Requisitos: RF 2.1 e RF 2.2\\

%Requisito 2.1
\end{tabular}
\end{center}
\end{table} 

\begin{center}
\def\arraystretch{1.1}
\begin{tabular}{|p{6cm}|p{6cm}|}

\hline
\textbf{Ação do Ator} & \textbf{Resposta do Sistema} \\ \hline
\textbf{1.} Este caso de uso tem inicio quando o Professor acessa a seção de administração do curso em questão  & \\ \hline
\textbf{2.} O Professor adiciona uma nova aula, informando descrição e conteúdo da mesma & \\ \hline
\textbf{3.} O Professor confirma a inserção da aula  & \textbf{4.} Registra e adiciona a nova aula ao curso\\ \hline
\end{tabular}
\end{center}

\textbf{Sequências Alternativas:} N/A

\newpage

%-------------------------------------------------------------------------------------------------------------
%Remover Conteúdos Inadequados
\begin{table}[h!]
\begin{center}
\begin{tabular}{p{2.5cm} p{9.5cm}}
Caso de uso: & \textbf{Remover Conteúdos Inadequados} \\ \hline
Atores: & Administrador \\ \hline
Finalidade: & Retirar conteúdos inadequados ou impróprios\\ \hline
Tipo: & Primário e Essencial \\ \hline
Visão geral: & O Administrador, após ser informado, de forma significativa, da existência de um conteúdo inadequado em um curso ou conta de usuário, o Administrador removerá a respectiva aula ou até a conta do usuário, e seus conteúdos relacionados. \\ \hline
Referências Cruzadas: & Requisitos: RF 3.1 e RF 3.2\\ 

%Requisito 3.2
\end{tabular}
\end{center}
\end{table}

\begin{center}
\def\arraystretch{1.1}
\begin{tabular}{|p{6cm}|p{6cm}|}

\hline
\textbf{Ação do Ator} & \textbf{Resposta do Sistema} \\ \hline
\textbf{1.} Este caso de uso tem inicio quando o Administrador identifica conteúdos inadequados  & \\ \hline
\textbf{2.} O Administrador remove o conteúdo inadequado informando o motivo da remoção do mesmo  & \textbf{3.} Remove o conteúdo em questão  \\ \hline
\end{tabular}
\end{center}

\textbf{Sequências Alternativas:} N/A

\newpage

%-------------------------------------------------------------------------------------------------------------
%Enviar Notificação de Advertência
\begin{table}[h!]
\begin{center}
\begin{tabular}{p{2.5cm} p{9.5cm}}
Caso de uso: & \textbf{Enviar Notificação de Advertência} \\ \hline
Atores: & Administrador \\ \hline
Finalidade: & Informar a um usuário que determinado conteúdo do mesmo é inadequado.\\ \hline
Tipo: & Primário e Essencial \\ \hline
Visão geral: & O Administrador, após ser informado, de forma significativa, que ocorreu um comportamento inadequado por parte de algum usuário, envia uma notificação de advertência para este último. \\ \hline
Referências Cruzadas: & Requisito: RF 3.3\\ 

%Requisito 3.3
\end{tabular}
\end{center}
\end{table}

\begin{center}
\def\arraystretch{1.1}
\begin{tabular}{|p{6cm}|p{6cm}|}

\hline
\textbf{Ação do Ator} & \textbf{Resposta do Sistema} \\ \hline
\textbf{1.} Este caso de uso tem inicio quando o Administrador identifica possiveis conteúdos inadequados adicionados por um Usuário  & \\ \hline
\textbf{2.} O Administrador envia uma notificação de advertência para o Usuário, que adicionou o conteúdo inadequado, informando a categoria e a descrição da advertência  & \textbf{3.} Envia notificação para o Usuário \\ \hline
\end{tabular}
\end{center}

\textbf{Sequências Alternativas:} N/A

\newpage

\begin{table}[h!]
\begin{center}
\begin{tabular}{p{2.5cm} p{9.5cm}}
Caso de uso: & \textbf{Acessar conteúdo de uma aula} \\ \hline
Atores: & Aluno \\ \hline
Finalidade: & Permite o acesso ao conteúdo de uma aula de um determinado curso. \\ \hline
Tipo: & Primário e Essencial\\ \hline
Visão geral: & O Aluno, após acessar a descrição do curso, seleciona-se a aula desejada e assim tem-se acesso ao seu respectivo conteúdo. \\ \hline
Referências Cruzadas: & Requisito: RF 1.18 \\ & Caso de uso: Listar Cursos e Usuários.

%Requisito 1.8
\end{tabular}
\end{center}
\end{table}

\begin{center}
\def\arraystretch{1.1}
\begin{tabular}{|p{6cm}|p{6cm}|}

\hline
\textbf{Ação do Ator} & \textbf{Resposta do Sistema} \\ \hline
\textbf{1.} O Aluno acessa a descrição do curso que contém a aula.  & \textbf{2.} Todas as aulas são listadas.\\ \hline
\textbf{3.} O Aluno seleciona a respectiva aula.  & \textbf{4.} O conteúdo da aula é exibido. \\ \hline
\end{tabular}
\end{center}

\textbf{Sequências Alternativas:} \\
\textbf{Linha 3:} Caso a aula desejada não esteja disponível, o aluno não poderá selecioná-la.

\newpage

\begin{table}[h!]
\begin{center}
\begin{tabular}{p{2.5cm} p{9.5cm}}
Caso de uso: & \textbf{Remover aula} \\ \hline
Atores: & Professor \\ \hline
Finalidade: & Possibilita a remoção de uma aula indesejada. \\ \hline
Tipo: & Primário e Essencial\\ \hline
Visão geral: & O Professor, após acessar a seção de administração do curso, seleciona a correspondente aula e a removê, deixando de fazer. \\ \hline
Referências Cruzadas: & Requisito: RF 2.3

%Requisito 2.3
\end{tabular}
\end{center}
\end{table}

\begin{center}
\def\arraystretch{1.1}
\begin{tabular}{|p{6cm}|p{6cm}|}

\hline
\textbf{Ação do Ator} & \textbf{Resposta do Sistema} \\ \hline
\textbf{1.} Este caso de uso tem inicio quando o Professor acessa a seção de administração do curso em questão. & \\ \hline
\textbf{2.} O Professor acessa a lista de aulas do curso e seleciona aquela a ser removida.  &  \textbf{3.} A aula é removida e uma confirmação é devolvida ao professor.\\ \hline

\end{tabular}
\end{center}

\textbf{Sequências Alternativas:} N/A

\newpage

\begin{table}[h!]
\begin{center}
\begin{tabular}{p{2.5cm} p{9.5cm}}
Caso de uso: & \textbf{Receber denúncia de usuário} \\ \hline
Atores: & Administrador \\ \hline
Finalidade: & Permite receber as denúncias de conteúdos inadequados dos usuários. \\ \hline
Tipo: & Primário e Essencial \\ \hline
Visão geral: & O Administrador, em seu perfil, recebe notificações de outros usuários, nas quais contém denúncias sobre conteúdos inadequados disponibilizados por outros usuários. \\ \hline
Referências Cruzadas: & Requisito: RF 3.4

%Requisito 3.4
\end{tabular}
\end{center}
\end{table}

\begin{center}
\def\arraystretch{1.1}
\begin{tabular}{|p{6cm}|p{6cm}|}

\hline
\textbf{Ação do Ator} & \textbf{Resposta do Sistema} \\ \hline
\textbf{1.} O Administrador acessa o seu perfil.  & \\ \hline
\textbf{2.} Caso haja a presença de denúncias cada situação é analisada e suas respectivas consequências.  & \\ \hline 
\end{tabular}
\end{center}

\textbf{Sequências Alternativas:} N/A

\end{document} %Fim do documento
