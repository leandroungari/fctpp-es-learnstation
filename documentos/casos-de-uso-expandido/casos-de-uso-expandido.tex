\documentclass[12pt,a4paper,onecolumn,titlepage]{article}

\usepackage[brazil]{babel}
%\usepackage[latin1]{inputenc}
\usepackage[utf8]{inputenc}
\usepackage{graphicx}

\begin{document} %Inicio do documento

\begin{titlepage} %Capa
	
	\vfill
	\begin{center}
	
		{\large \textbf{Faculdade de Ciências e Tecnologia\\Universidade Estadual Paulista\\``Júlio de Mesquita Filho''}} \\[3cm]
		{\large \textbf{Bruno Santos de Lima}}\\
		{\large \textbf{Leandro Ungari Cayres}}\\[4cm]
		{\Large Casos de Uso Expandido}\\
		{\Large Sistema LearnStation}\\[4cm]

	\hspace{.45\textwidth} %posiciona a minipage
	\begin{minipage}{.5\textwidth}
		\large Disciplina de Engenharia de Software I. Professor Dr. Rogério Eduardo Garcia.\\[0.5cm]
	\end{minipage}

	\vfill
	\vspace{1.5cm}
	
	\large \textbf{Presidente Prudente\\}
	\large \textbf{Maio - 2016}
	
	\end{center}
	
\end{titlepage}
%Fim da capa

%Conteúdo do documento
\renewcommand{\baselinestretch}{1.5}
%\def\arraystretch{1.1}

\newpage

%-------------------------------------------------------------------------------------------------------------
%Avaliação de curso
\begin{table}[h!]
\begin{center}
\begin{tabular}{p{2.5cm} p{9.5cm}}
Caso de uso: & \textbf{Avaliação de curso} \\ \hline
Atores: & Aluno \\ \hline
Finalidade: & Capturar a avaliação de um curso\\ \hline
Tipo: & Essencial \\ \hline
Visão geral: & O Aluno inscrito no curso em questão acessa este por meio da sua lista de cursos inscritos, ou por meio da busca de cursos, em seguida, avalia este atribuindo uma nota. \\ \hline
Referências Cruzadas: & Requisitos: R1.6, R1.12\\

%Requisito 1.12
\end{tabular}
\end{center}
\end{table} 

\begin{center}
\def\arraystretch{1.1}
\begin{tabular}{|p{6cm}|p{6cm}|}

\hline
\textbf{Ação do Ator} & \textbf{Resposta do Sistema} \\ \hline
\textbf{1.} Este caso de uso tem inicio quando o Aluno acessa a descrição de um determinado curso  & \textbf{2.} Exibe a descrição do curso para o Aluno  \\ \hline
\textbf{3.} O Aluno acessa a seção de avaliação do curso em questão.  & \textbf{4.} Exibe a seção de avaliação do curso, descrevendo a nota atual deste curso e permitindo ao Aluno atibuir uma nota ao curso em questão  \\ \hline
\textbf{5.} O Aluno atribui uma nota para o curso em questão  &   \\ \hline
\textbf{6.} O Aluno finaliza a avaliação do curso  & \textbf{7.} A nota do Aluno é resgistada e é calculada um nova nota geral do curso avalidado \\ \hline
\end{tabular}
\end{center}

\textbf{Sequências Alternativas:} N/A


\newpage

%-------------------------------------------------------------------------------------------------------------
%Enviar mensagem para um amigo
\begin{table}[h!]
\begin{center}
\begin{tabular}{p{2.5cm} p{9.5cm}}
Caso de uso: & \textbf{Enviar mensagem para um amigo} \\ \hline
Atores: & Remetente, Destinatário \\ \hline
Finalidade: & Registrar e enviar mensagem de um remetente para um destinatário\\ \hline
Tipo: & Essencial \\ \hline
Visão geral: & O Usuário por meio de uma lista de amigos, selecione um referido amigo e envia mensagens textuais através de uma seção de conversa. \\ \hline
Referências Cruzadas: & Requisitos: R1.2, R1.12\\

%Requisito 1.12
\end{tabular}
\end{center}
\end{table} 

\begin{center}
\def\arraystretch{1.1}
\begin{tabular}{|p{6cm}|p{6cm}|}

\hline
\textbf{Ação do Ator} & \textbf{Resposta do Sistema} \\ \hline
\textbf{1.} Este caso de uso tem inicio quando o Remetente seleciona um Destinatário em sua lista de amigos para iniciar uma conversa  &  \\ \hline
\textbf{2.} O Remetente escreve a mensagem  &   \\ \hline
\textbf{3.} O Remetente envia a mensagem para o Destinatário  &
\textbf{4.} Envia a mensagem do Remetente e também envia uma notificação de uma nova mensagem recebida para o Destinatário \\ \hline
\end{tabular}
\end{center}

\textbf{Sequências Alternativas:} N/A

\newpage

%-------------------------------------------------------------------------------------------------------------
%Receber mensagem de um amigo
\begin{table}[h!]
\begin{center}
\begin{tabular}{p{2.5cm} p{9.5cm}}
Caso de uso: & \textbf{Receber mensagem de um amigo} \\ \hline
Atores: & Destinatário, Remetente \\ \hline
Finalidade: & Notificar e encaminhar mensagem para o destinatário\\ \hline
Tipo: & Essencial \\ \hline
Visão geral: & O Usuário acessa sua conta, verifica a existência de notificações, no caso, de recebimento de mensagens, em seguida, seleciona uma determinada notificação tendo acesso ao conteúdo da mensagem. \\ \hline
Referências Cruzadas: & Requisitos: R1.12, R1.13\\

%Requisito 1.12, 1.13
\end{tabular}
\end{center}
\end{table} 

\begin{center}
\def\arraystretch{1.1}
\begin{tabular}{|p{6cm}|p{6cm}|}

\hline
\textbf{Ação do Ator} & \textbf{Resposta do Sistema} \\ \hline
\textbf{1.} Este caso de uso tem inicio quando o Destinatário verifica suas notificações de mensagens enviadas  &  \\ \hline
\textbf{2.} O destinatário acessa o conteúdo desta mensagem  & 
\textbf{3.} Informa ao destinatário o conteúdo da mensagem no qual ele recebeu \\ \hline
\textbf{4.} Destinatário lê o conteúdo da mensagem recebida. & \\ \hline
\end{tabular}
\end{center}

\textbf{Sequências Alternativas:} N/A

\newpage

%-------------------------------------------------------------------------------------------------------------
%Encerrar conta de usuário
\begin{table}[h!]
\begin{center}
\begin{tabular}{p{2.5cm} p{9.5cm}}
Caso de uso: & \textbf{Encerrar conta de usuário} \\ \hline
Atores: & Usuário \\ \hline
Finalidade: & Finalizar e encerrar um conta de usuário\\ \hline
Tipo: & Essencial \\ \hline
Visão geral: & O Usuário acessa sua conta, em seguida, solicita o cancelamento da sua conta fornecendo informação de segurança. \\ \hline
Referências Cruzadas: & Requisitos: R1.14, R1.15\\

%Requisito 1.14, 1.15
\end{tabular}
\end{center}
\end{table} 

\begin{center}
\def\arraystretch{1.1}
\begin{tabular}{|p{6cm}|p{6cm}|}

\hline
\textbf{Ação do Ator} & \textbf{Resposta do Sistema} \\ \hline
\textbf{1.} Este caso de uso tem inicio quando o Usuário acessa seu perfil e solicita a exclusão de sua conta  & \textbf{2.} Inicia o processo de exlcusão de conta de Usuário pedindo ao mesmo que confime o pedido de exclusão de conta \\ \hline
\textbf{3.} O Usuário confirma o pedido de exclusão de conta  & \textbf{4.} Pede para o Usuário informar a senha da conta em questão  \\ \hline
\textbf{5.} O Usuário informa a senha de sua conta  & \textbf{6.} Verifica se a senha está correta e exclui a conta do Usuário\\ \hline
\end{tabular}
\end{center}

\textbf{Sequências Alternativas:} \\
\textbf{Linha 6:} Ao verificar que a senha fornecida pelo Usuário está incorreta, informa o mesmo sobre a senha incorreta e não exclui a conta do Usuário.

\newpage

%-------------------------------------------------------------------------------------------------------------
%Denunciar Conteúdo Inadequado
\begin{table}[h!]
\begin{center}
\begin{tabular}{p{2.5cm} p{9.5cm}}
Caso de uso: & \textbf{Denunciar Conteúdo Inadequado} \\ \hline
Atores: & Usuário, Curso \\ \hline
Finalidade: & Permitir e registrar denuncia de conteúdo inadequado\\ \hline
Tipo: & Essencial \\ \hline
Visão geral: & O Usuário ao acessar o perfil de um usuário ou a descrição de um curso, ao perceber a existência de conteúdo inadequado realizará, o usuário poderá denunciar ao administrador do sistema. \\ \hline
Referências Cruzadas: & Requisitos: R1.3, R1.16, R2.2\\

%Requisito 1.16
\end{tabular}
\end{center}
\end{table} 


\begin{center}
\def\arraystretch{1.1}
\begin{tabular}{|p{6cm}|p{6cm}|}

\hline
\textbf{Ação do Ator} & \textbf{Resposta do Sistema} \\ \hline
\textbf{1.} Este caso de uso tem inicio quando o Usuário identifica um conteúdo inadequado & \\ \hline
\textbf{2.} O Usuário faz a denuncia do conteúdo  & \textbf{3.} Pede para o Usuário a categoria e a descrição sobre o conteúdo inadequado  \\ \hline
\textbf{4.} O Usuário informa a categoria e a descrição do conteúdo inadequado & \\ \hline
\textbf{5.} O Usuário envia a denuncia do conteúdo Inadequado & \textbf{6} Regrista a denuncia \\ \hline
\end{tabular}
\end{center}

\textbf{Sequências Alternativas:} N/A

\newpage

%-------------------------------------------------------------------------------------------------------------
%Adicionar Aula
\begin{table}[h!]
\begin{center}
\begin{tabular}{p{2.5cm} p{9.5cm}}
Caso de uso: & \textbf{Adicionar Aula} \\ \hline
Atores: & Professor \\ \hline
Finalidade: & Registrar nova aula inserida por um professor em um determinado curso\\ \hline
Tipo: & Essencial \\ \hline
Visão geral: & O Professor acessa a seção de administração de um curso, adiciona uma nova aula fornecendo seu respectivo conteúdo. \\ \hline
Referências Cruzadas: & Requisitos: R2.1, R2.2\\

%Requisito 2.1
\end{tabular}
\end{center}
\end{table} 

\begin{center}
\def\arraystretch{1.1}
\begin{tabular}{|p{6cm}|p{6cm}|}

\hline
\textbf{Ação do Ator} & \textbf{Resposta do Sistema} \\ \hline
\textbf{1.} Este caso de uso tem inicio quando o Professor acessa a seção de administração do curso em questão  & \\ \hline
\textbf{2.} O Professor adiciona uma nova aula  & \textbf{3.} Recebe a aula e pede para o Professor confirmar a inserção da mesma  \\ \hline
\textbf{4.} O Professor confirma a inserção da aula  & \textbf{5.} Adiciona a nova aula ao curso\\ \hline
\end{tabular}
\end{center}

\textbf{Sequências Alternativas:} N/A

\newpage

%-------------------------------------------------------------------------------------------------------------
%Remover Conteúdos Inadequados
\begin{table}[h!]
\begin{center}
\begin{tabular}{p{2.5cm} p{9.5cm}}
Caso de uso: & \textbf{Remover Conteúdos Inadequados} \\ \hline
Atores: & Administrador \\ \hline
Finalidade: & Retirar conteúdos inadequados ou impróprios\\ \hline
Tipo: & Essencial \\ \hline
Visão geral: & O Administrador, após ser informado, de forma significativa, da existência de um conteúdo inadequado em um curso ou conta de usuário, o Administrador removerá a respectiva aula ou até a conta do usuário, e seus conteúdos relacionados. \\ \hline
Referências Cruzadas: & Requisitos: R3.1, R3.2\\ 

%Requisito 3.2
\end{tabular}
\end{center}
\end{table}

\begin{center}
\def\arraystretch{1.1}
\begin{tabular}{|p{6cm}|p{6cm}|}

\hline
\textbf{Ação do Ator} & \textbf{Resposta do Sistema} \\ \hline
\textbf{1.} Este caso de uso tem inicio quando o Administrador identifica conteúdos inadequados  & \\ \hline
\textbf{2.} O Administrador remove o conteúdo inadequado  & \textbf{3.} Remove o conteúdo em questão  \\ \hline
\end{tabular}
\end{center}

\textbf{Sequências Alternativas:} N/A

\newpage

%-------------------------------------------------------------------------------------------------------------
%Enviar Notificação de Advertência
\begin{table}[h!]
\begin{center}
\begin{tabular}{p{2.5cm} p{9.5cm}}
Caso de uso: & \textbf{Enviar Notificação de Advertência} \\ \hline
Atores: & Administrador \\ \hline
Finalidade: & Informar a um usuário que determinado conteúdo do mesmo é inadequado.\\ \hline
Tipo: & Essencial \\ \hline
Visão geral: & O Administrador, após ser informado, de forma significativa, que ocorreu um comportamento inadequado por parte de algum usuário, envia uma notificação de advertência para este último. \\ \hline
Referências Cruzadas: & Requisitos: R3.3\\ 

%Requisito 3.3
\end{tabular}
\end{center}
\end{table}

\begin{center}
\def\arraystretch{1.1}
\begin{tabular}{|p{6cm}|p{6cm}|}

\hline
\textbf{Ação do Ator} & \textbf{Resposta do Sistema} \\ \hline
\textbf{1.} Este caso de uso tem inicio quando o Administrador identifica possiveis conteúdos inadequados adicionados por um Usuário  & \\ \hline
\textbf{2.} O Administrador envia uma notificação de advertência para o Usuário que adicionou o conteúdo inadequado  & \textbf{3.} Envia notificação para o Usuário \\ \hline
\end{tabular}
\end{center}

\textbf{Sequências Alternativas:} N/A

\end{document} %Fim do documento
