\documentclass[12pt,a4paper,onecolumn,titlepage]{article}

\usepackage[brazil]{babel}
%\usepackage[latin1]{inputenc}
\usepackage[utf8]{inputenc}
\usepackage{graphicx}

\begin{document} %Inicio do documento

\begin{titlepage} %Capa
	
	\vfill
	\begin{center}
	
		{\large \textbf{Faculdade de Ciências e Tecnologia\\Universidade Estadual Paulista\\``Júlio de Mesquita Filho''}} \\[3cm]
		{\large \textbf{Bruno Santos de Lima}}\\
		{\large \textbf{Leandro Ungari Cayres}}\\[4cm]
		{\Large Casos de Uso}\\
		{\Large Sistema LearnStation}\\[4cm]

	\hspace{.45\textwidth} %posiciona a minipage
	\begin{minipage}{.5\textwidth}
		\large Disciplina de Engenharia de Software I. Professor Dr. Rogério Eduardo Garcia.\\[0.5cm]
	\end{minipage}

	\vfill
	\vspace{1.5cm}
	
	\large \textbf{Presidente Prudente\\}
	\large \textbf{Abril - 2016}
	
	\end{center}
	
\end{titlepage}
%Fim da capa

%Conteúdo do documento
\renewcommand{\baselinestretch}{1.5}

\begin{table}[h!]
\begin{center}
\begin{tabular}{p{2.5cm} p{9.5cm}}
Caso de uso: & \textbf{Criar Conta de Usuário} \\ \hline
Atores: & Usuário \\ \hline
Tipo: & Primário \\ \hline
Descrição: & O Usuário acessa a seção de cadastro fornecendo informações pessoais. Em seguida, os dados são validados e o cadastro concluído.

%Requisito 1.1	
\end{tabular}
\end{center}
\end{table}

\begin{table}[h!]
\begin{center}
\begin{tabular}{p{2.5cm} p{9.5cm}}
Caso de uso: & \textbf{Iniciar sessão de usuário} \\ \hline
Atores: & Usuário \\ \hline
Tipo: & Primário \\ \hline
Descrição: & O Usuário acessa a seção de entrada do sistema, fornece os dados de entrada que são validados pelo sistema e em seguida tem acesso ao seu perfil.

%Requisito 1.17	
\end{tabular}
\end{center}
\end{table}


\begin{table}[h!]
\begin{center}
\begin{tabular}{p{2.5cm} p{9.5cm}}
Caso de uso: & \textbf{Finalizar sessão de usuário} \\ \hline
Atores: & Usuário \\ \hline
Tipo: & Primário \\ \hline
Descrição: & O Usuário dentro do sistema, pode aciona a finalização de sua sessão através de qualquer página.

%Requisito 1.17	
\end{tabular}
\end{center}
\end{table}

\clearpage

\begin{table}[h!]
\begin{center}
\begin{tabular}{p{2.5cm} p{9.5cm}}
Caso de uso: & \textbf{Listar Cursos e Usuários} \\ \hline
Atores: & Usuário \\ \hline
Tipo: & Primário \\ \hline
Descrição: & O Usuário, por meio do mecanismo de pesquisa, de forma textual busca cursos ou usuários através de seus respectivos nome de usuário ou curso, e área de conhecimento para cursos. Desta forma, tendo acesso ao conteúdo do perfil do usuário ou curso buscado.

%Requisito 1.2, 1.3, 1.6, 2.2
\end{tabular}
\end{center}
\end{table}


\begin{table}[h!]
\begin{center}
\begin{tabular}{p{2.5cm} p{9.5cm}}
Caso de uso: & \textbf{Enviar Solicitação de Amizade} \\ \hline
Atores: & Usuário \\ \hline
Tipo: & Primário \\ \hline
Descrição: & O Usuário busca pelo referido usuário, em seguida acessa o perfil deste e envia uma solicitação de amizade.

%Requisito 1.4
\end{tabular}
\end{center}
\end{table}

\clearpage

\begin{table}[h!]
\begin{center}
\begin{tabular}{p{2.5cm} p{9.5cm}}
Caso de uso: & \textbf{Responder Solicitação} \\ \hline
Atores: & Usuário \\ \hline
Tipo: & Primário \\ \hline
Descrição: & O Usuário após acessar sua conta, verifica a existência de uma notificação, a qual é uma solicitação de amizade, em seguida, se desejado o Usuário poderá confirmar, recusar ou ignorar.

%Requisito 1.4, 1.13
\end{tabular}
\end{center}
\end{table}


\begin{table}[h!]
\begin{center}
\begin{tabular}{p{2.5cm} p{9.5cm}}
Caso de uso: & \textbf{Cancelar amizade} \\ \hline
Atores: & Usuário \\ \hline
Tipo: & Primário \\ \hline
Descrição: & O Usuário realiza uma busca pelo referido amigo, ou acessa a lista de amizades, em seguida, visita o perfil do determinado amigo, efetua o cancelamento desta amizade.

%Requisito 1.5
\end{tabular}
\end{center}
\end{table}



\begin{table}[h!]
\begin{center}
\begin{tabular}{p{2.5cm} p{9.5cm}}
Caso de uso: & \textbf{Efetuar Inscrição em Curso} \\ \hline
Atores: & Aluno \\ \hline
Tipo: & Primário \\ \hline
Descrição: & Após a realização da busca pelo referido curso e visualizar da descrição do curso, o Aluno realiza a inscrição neste curso.

%Requisito 1.7
\end{tabular}
\end{center}
\end{table}


\begin{table}[h!]
\begin{center}
\begin{tabular}{p{2.5cm} p{9.5cm}}
Caso de uso: & \textbf{Cancelar Inscrição em Curso} \\ \hline
Atores: & Aluno \\ \hline
Tipo: & Primário \\ \hline
Descrição: & O Aluno realiza uma busca pelo curso, ou acessa a lista de cursos em que está inscrito, em seguida, acessa a descrição deste e realiza o cancelamento da sua inscrição.

%Requisito 1.8
\end{tabular}
\end{center}
\end{table}


\begin{table}[h!]
\begin{center}
\begin{tabular}{p{2.5cm} p{9.5cm}}
Caso de uso: & \textbf{Criar Novo Curso} \\ \hline
Atores: & Professor \\ \hline
Tipo: & Primário \\ \hline
Descrição: & O Professor acessa a seção de cadastro de curso, fornecendo as informações sobre o curso que será criado e finaliza a operação.

%Requisito 1.9
\end{tabular}
\end{center}
\end{table}


\begin{table}[h!]
\begin{center}
\begin{tabular}{p{2.5cm} p{9.5cm}}
Caso de uso: & \textbf{Avaliar Curso} \\ \hline
Atores: & Aluno \\ \hline
Tipo: & Primário \\ \hline
Descrição: & O Aluno inscrito no curso em questão acessa este por meio da sua lista de cursos inscritos, ou por meio da busca de cursos, em seguida, avalia este atribuindo uma nota.

%Requisito 1.10
\end{tabular}
\end{center}
\end{table}


\begin{table}[h!]
\begin{center}
\begin{tabular}{p{2.5cm} p{9.5cm}}
Caso de uso: & \textbf{Enviar Mensagem para Amigo} \\ \hline
Atores: & Remetente, Destinatário \\ \hline
Tipo: & Primário \\ \hline
Descrição: & O Usuário por meio de uma lista de amigos, selecione um referido amigo e envia mensagens textuais através de uma seção de conversa.

%Requisito 1.12
\end{tabular}
\end{center}
\end{table}


\begin{table}[h!]
\begin{center}
\begin{tabular}{p{2.5cm} p{9.5cm}}
Caso de uso: & \textbf{Receber Mensagem de Amigo} \\ \hline
Atores: & Destinatário, Remetente \\ \hline
Tipo: & Primário \\ \hline
Descrição: & O Usuário acessa sua conta, verifica a existência de notificações, no caso, de recebimento de mensagens, em seguida, seleciona uma determinada notificação tendo acesso ao conteúdo da mensagem.

%Requisito 1.12, 1.13
\end{tabular}
\end{center}
\end{table}


\begin{table}[h!]
\begin{center}
\begin{tabular}{p{2.5cm} p{9.5cm}}
Caso de uso: & \textbf{Encerrar Conta de Usuário} \\ \hline
Atores: & Usuário \\ \hline
Tipo: & Primário \\ \hline
Descrição: & O Usuário acessa sua conta, em seguida, solicita o cancelamento da sua conta fornecendo informação de segurança.

%Requisito 1.14, 1.15
\end{tabular}
\end{center}
\end{table}


\begin{table}[h!]
\begin{center}
\begin{tabular}{p{2.5cm} p{9.5cm}}
Caso de uso: & \textbf{Denunciar Conteúdo Inadequado} \\ \hline
Atores: & Usuário \\ \hline
Tipo: & Primário \\ \hline
Descrição: & O Usuário ao acessar o perfil de um usuário ou a descrição de um curso, ao perceber a existência de conteúdo inadequado realizará, o usuário poderá denunciar ao administrador do sistema.

%Requisito 1.16
\end{tabular}
\end{center}
\end{table}

\begin{table}[h!]
\begin{center}
\begin{tabular}{p{2.5cm} p{9.5cm}}
Caso de uso: & \textbf{Adicionar Aula} \\ \hline
Atores: & Professor \\ \hline
Tipo: & Primário \\ \hline
Descrição: & O Professor acessa a seção de administração de um curso, adiciona uma nova aula fornecendo seu respectivo conteúdo.

%Requisito 2.1
\end{tabular}
\end{center}
\end{table}


\begin{table}[h!]
\begin{center}
\begin{tabular}{p{2.5cm} p{9.5cm}}
Caso de uso: & \textbf{Listar Cursos e Usuários para Administração} \\ \hline
Atores: & Administrador \\ \hline
Tipo: & Primário \\ \hline
Descrição: & O Administrador realiza uma consulta de contas de usuário, aulas ou cursos.

%Requisito 3.1
\end{tabular}
\end{center}
\end{table}


\begin{table}[h!]
\begin{center}
\begin{tabular}{p{2.5cm} p{9.5cm}}
Caso de uso: & \textbf{Remover Conteúdos Inadequados} \\ \hline
Atores: & Administrador \\ \hline
Tipo: & Primário \\ \hline
Descrição: & O Administrador, após ser informado, de forma significativa, da existência de um conteúdo inadequado em um curso ou conta de usuário, o Administrador removerá a respectiva aula ou até a conta do usuário, e seus conteúdos relacionados.

%Requisito 3.2
\end{tabular}
\end{center}
\end{table}

\begin{table}[h!]
\begin{center}
\begin{tabular}{p{2.5cm} p{9.5cm}}
Caso de uso: & \textbf{Enviar Notificação de Advertência} \\ \hline
Atores: & Administrador \\ \hline
Tipo: & Primário \\ \hline
Descrição: & O Administrador, após ser informado, de forma significativa, o Administrador envia uma notificação de advertência devido a algum comportamento.

%Requisito 3.3
\end{tabular}
\end{center}
\end{table}

%acessar conteudo de uma aula 1.18
%remover uma aula de um determinado curso 2.3
%receber denuncias de usuário 3.4

\begin{table}[h!]
\begin{center}
\begin{tabular}{p{2.5cm} p{9.5cm}}
Caso de uso: & \textbf{Acessar conteúdo de uma aula} \\ \hline
Atores: & Aluno \\ \hline
Tipo: & Primário \\ \hline
Descrição: & O Aluno, após acessar a descrição do curso, seleciona-se a aula desejada e assim tem-se acesso ao seu respectivo conteúdo.

%Requisito 1.18
\end{tabular}
\end{center}
\end{table}

\begin{table}[h!]
\begin{center}
\begin{tabular}{p{2.5cm} p{9.5cm}}
Caso de uso: & \textbf{Remover aula} \\ \hline
Atores: & Professor \\ \hline
Tipo: & Primário \\ \hline
Descrição: & O Professor, após acessar a seção de administração do curso, seleciona a correspondente aula e a removê, deixando de fazer.

%Requisito 2.3
\end{tabular}
\end{center}
\end{table}

\begin{table}[h!]
\begin{center}
\begin{tabular}{p{2.5cm} p{9.5cm}}
Caso de uso: & \textbf{Receber denúncia de usuário} \\ \hline
Atores: & Administrador \\ \hline
Tipo: & Primário \\ \hline
Descrição: & O Administrador, em seu perfil, recebe notificações de outros usuários, nas quais contém denúncias sobre conteúdos inadequados disponibilizados por outros usuários.

%Requisito 3.4
\end{tabular}
\end{center}
\end{table}

\end{document} %Fim do documento
